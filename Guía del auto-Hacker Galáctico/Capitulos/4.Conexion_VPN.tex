\newpage
\section{Conexión a la VPN, \textsc{HackTheCERN / HackTheCSN / HackTheUCM}}
\subsection{Código ético}
Antes de comenzar, os adjuntamos el \textit{Manual de Código Ético} proporcionado por el \textbf{CERN} y por el \textbf{CSN}. Es muy recomendable que les echéis un vistazo, pues explica las direcciones IPs a las que debéis enfocar vuestros ataques y cómo debéis realizar estos mismos:
\begin{tcolorbox}
    \url{https://cv4.ucm.es/moodle/pluginfile.php/4672494/mod_resource/content/1/CERNCodeOfEthics.pdf}
\end{tcolorbox} 

\begin{tcolorbox}
    \url{https://cv4.ucm.es/moodle/pluginfile.php/7094461/mod_resource/content/1/CodigoEtico.pdf}
\end{tcolorbox} 

\subsection{Registro}
Para poder tener acceso a la VPN, \textbf{es necesario que rellenéis este formulario} con vuestra cuenta UCM y esperéis una confirmación mediante correo electrónico de que se os ha concedido acceso, \textbf{hasta entonces, no podréis realizar ningún ataque}.
\begin{tcolorbox}
    \url{https://fdist.ucm.es/registro/}
\end{tcolorbox}

\subsection{Configuración de la VPN}
\subsubsection{En distribuciones Linux} 
\begin{tcolorbox}[colframe=black!75!blue]
\begin{center}
    (Basadas en Debian: Ubuntu, Linux Mint, Raspbian...)
\end{center}
\end{tcolorbox}
\begin{enumerate}
    \item Abrimos un terminal y escribimos el siguiente comando: 
        \begin{center}
         \textbf{sudo apt-get install vpnc network-manager-vpnc} 
        \end{center}
        \item Si tenemos un escritorio basado en GNOME, escribiremos: 
        \begin{center}
         \textbf{sudo apt-get install network-manager-vpnc-gnome}
        \end{center}
\end{enumerate}

\begin{tcolorbox}[colframe=black!75!red]
\begin{center}
    (En otras versiones como RedHat, Arch, Manjaro...)
\end{center}
\end{tcolorbox}
\begin{enumerate}
    \item Abrimos un terminal y escribimos el siguiente comando: 
        \begin{center}
         \textbf{yum  install vpnc} 
        \end{center}
        \item Si no funciona, escribiremos: 
        \begin{center}
         \textbf{yum install NetworkManager‐vpnc}
        \end{center}
\end{enumerate}

\newline \textsc{(Esto instalará los paquetes para servidor VPN)}.

\begin{itemize}
    \item Una vez instalados todos los paquetes, escribiremos el comando:
    \begin{center}
         \textbf{sudo vpnc} 
    \end{center}
    
    \item A continuación nos pedirá los credenciales, debemos escribir lo siguiente:
    \begin{tcolorbox}
        \begin{center}
            \begin{itemize}
               \item Pasarela / Gateway: \textbf{reservado2.vpn.ucm.es} 
               \item Nombre del grupo: \textbf{ucm}
               \item Contraseña del grupo: \textbf{ucm}
               \item Nombre de usuario: \textbf{Vuestro correo @ucm.es}
               \item Contraseña del usuario: \textbf{Vuestra contraseña del correo}
             \end{itemize}
        \end{center}
    \end{tcolorbox}
    
    \item Una vez que queramos desconectarnos, debemos escribir el siguiente comando:
        \begin{center}
            \textbf{sudo vpnc-disconnect}
        \end{center}
        \newline (O simplemente reiniciar el equipo).
\end{itemize}

\subsubsection{En Windows 10}
Si estamos usando una máquina virtual de Linux en Windows 10 y conectamos Windows a la VPN, la máquina virtual (que está soportada por Windows 10 como SO principal), también estará conectada a la VPN.
\begin{itemize}
    \item En primer lugar, es necesario bajar e instalar el programa \textit{Global Protect} desde la página:
    \begin{center}
        \url{https://galeria.ucm.es/global-protect/login.esp}
    \end{center}
    
    \item  Su instalación es muy sencilla y una vez instalado introducimos las credenciales:

 \begin{tcolorbox}
        \begin{center}
            \begin{itemize}
               \item Pasarela / Gateway: \textbf{reservado2.vpn.ucm.es} 
               \item Nombre de usuario: \textbf{Vuestro correo @ucm.es}
               \item Contraseña del usuario: \textbf{Vuestra contraseña del correo}
             \end{itemize}
        \end{center}
    \end{tcolorbox}
    
    \item Y cuando queramos cerrar la VPN, es tan sencillo como pulsar en \textit{'Desconectar'}.
\end{itemize}




