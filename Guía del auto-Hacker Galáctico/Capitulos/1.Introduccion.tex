\newpage
\section{Introducción}
Bienvenidos a la \testxtbf{Guía del auto-Hacker Galáctico}, un breve documento en el que se explicarán las nociones básicas para comenzar tu camino como \textit{hacker} en \textbf{FDIst}.
\newline
\noindent
\subsection{Sistemas Operativos}
\noindent
He de comenzar diciendo que todo lo que se explique aquí, irá enfocado a la plataforma \testxtbf{Linux}, sin embargo, si usas \textbf{Windows}, no te preocupes, pues te enseñaremos a crear y usar una máquina virtual para que puedas trabajar sin necesidad de tener ninguna distribución de \textbf{GNU/Linux}.
\newline\\
Además, recalcar que cada uno es libre de usar la plataforma en la que más cómodo se sienta, no por usar \textbf{Windows o MacOS}, serás un informático de segunda, las herramientas están ahí y gracias a internet, toda la documentación está a vuestro alcance. Por tanto, si preferís usar otro Sistema Operativo (SO), que no esté basado en GNU/Linux, sabed que tenéis herramientas equivalentes a las que se van a detallar en esta guía. Eso sí, \textbf{tened cuidado con qué instaláis}, pues existen programas de dudosa seguridad y que incluyen spyware.
\subsection{Advertencia y VPN}
\noindent
Por último advertíos de que \textbf{FDIst} tiene un acuerdo con el \textbf{CERN}, el \textbf{CSN} y el \textbf{Centro de Procesamiento de Datos (CPD)} de la Universidad Complutense de Madrid (UCM). Cualquier ataque a otra entidad o empresa, no estará respaldado por FDIst y \textbf{todo ataque que se realice a los centros anteriormente mencionados, deberá ser a través de la VPN} que se os proporcionará en este mismo manual y a la que previamente se os habrá dado acceso a través de vuestro correo UCM.