\newpage
\section{Estrategias.}
\noindent
\textsc{Siempre se halla la eterna duda entre \textbf{FUERZA} y \textbf{VELOCIDAD}.
\newline Rellenar rápidamente el núcleo o atacar a posiciones estratégicas.} \\
\\ En CoreWar existen \textbf{3 estrategias} fundamentales que, por analogía con el famoso juego, se les denominan \textbf{piedra, papel y tijera}.

\begin{tcolorbox}[colframe=black!75!blue]
    \begin{enumerate}
        \item \textbf{PAPEL: } \underline{hace múltiples copias de sí mismo lo más rápidamente posible}, así sacrifica velocidad de ataque por resistencia.
        \newline Esta estrategia \underline{vence a piedra} pero \underline{pierde ante tijeras} gracias a su gran capacidad de supervivencia aunque tienen una cierta tendencia al empate. 
        \item  \textbf{PIEDRA: } \underline{bombardea direcciones de memoria a ciegas} intentando matar rápidamente al mayor número de enemigos. Su reducido tamaño y sencillez los hace relativamente robustos y difíciles de localizar.
        \newline Esta estrategia \underline{vence a tijeras} pero \underline{pierde ante papel}. 
         \item \textbf{TIJERA: } \underline{es la más avanzada.} Comprueba posiciones de memoria a intervalos hasta localizar al guerrero rival. Una vez localizado generalmente sobrescriben su código con instrucciones que les obligan a generar nuevos procesos indefinidamente hasta quedar prácticamente bloqueados. Después proceden a eliminar todos los rivales.
         \newline Esta estrategia generalmente \underline{vence al papel} y \underline{pierde contra piedra}, puesto que pierde tiempo atacando las posiciones de memoria alteradas por este último. 
    \end{enumerate}
\end{tcolorbox}